PLAN

Intro
1. La/Les  structure(s)
  1.1 nos choix techniques
  1.2 difficultés rencontrées
  1.3 améliorations possibles
  1.4 erreurs faites et limites
2.Création Labyrinthe :
  2.1. Traitement fichier//Lucie
    2.1.1 lecture dans un fichier
    2.1.2 Sauvegarde dans un fichier
  2.2. Génération Aléatoire//Fatma
  2.3 Validite du labyrinthe //Lucie
3.Chemin //Morgane
  3.1 Chemin...
4. Affichage //Moe
  4.1 avec chemin/sans chemin a toi de voir si tu fais deux parties disctinsctes moe ;)
5. Main et Makefile//Lucie
  5.1 Main
    5.1.1
    5.1.2
6.Exemples d'executions
  6.1 Test en dur
  6.2 Lecture fichier
  6.3 Aleatoire
    6.3.1 PSeudo
    6.3.2 Totalement
7.Commentaires Globals
Conclu






Intro
Choix techniques /Difficultés rencontrées/erreurs faites/limites des algos


0. La/Les  structure(s)
0.1 nos choix techniques
 
0.2 difficultés rencontrées

0.3 améliorations possibles
0.4 erreurs faites et limites

2.Création Labyrinthe :

2.1. Traitement fichier//Lucie
2.1.1 lecture dans un fichier
2.1.2 Sauvegarde dans un fichier

2.2. Génération Aléatoire//Fatma

3.Chemin //Morgane
3.1 Chemin...

4. Affichage 
4.1 avec chemin/sans chemin a toi de voir si tu fais deux parties disctinsctes moe ;)

5. Main et Makefile
5.1 Main
5.1.1



5.2 Makefile
5.2.1 Pourquoi en faire un
Le principe du Makefile est de découper son code en plusieurs morceaux distincts de façon à le rendre plus compréhensible.
C'est également plus facile de travailler en groupe de la sorte. 
Au vu de l'envergure du projet, et vu le fait que nous étions au nombre de quatre, il était impératif d'en faire un. 
De sorte, il y a au total un main.c, qui est le main général. Un fichier .h qui est inclus dans tous les autres fichiers .c.  
Et sept fichiers.c distincts, validite.c, fichiers.c, affichage.c, mazerand1.c, mazerand2.c, mazepile.c et waysearch.c.
Tous ces fichiers sont donc liés par le Makefile ce qui permet de distinguer leur utilité et de les modifier plus facilement au besoin.

5.2.2 La mise en pratique et les choix techniques
J'ai décidé en faisant le Makefile, de ne pas alourdir l'écriture avec les différentes commandes possibles.
J'ai simplement rajouté un "clean" pour supprimer les fichiers .o générés lors de la compilation. 



6.Exemples d'executions
6.1 Test en dur
6.2 Lecture fichier
6.3 Aleatoire
6.3.1 PSeudo
6.3.2 Totalement

7.


.
.
.

Conclu
